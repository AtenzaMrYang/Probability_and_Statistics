\section{概率论的基本概念}

\begin{question}{题目2}
    设 $A,B,C$ 为三个事件,用 $A,B,C$ 的运算关系表示下列各事件:
    \begin{itemize}
        \item[(1)] $A$ 发生,$B$ 与 $C$ 不发生.
        \item[(2)] $A$ 与 $B$都发生,而 $C$ 不发生.
        \item[(3)] $A,B,C$ 中至少有一个发生.
        \item[(4)] $A,B,C$ 都发生.
        \item[(5)] $A,B,C$ 都不发生.
        \item[(6)] $A,B,C$ 中不多于一个发生.
        \item[(7)] $A,B,C$ 中不多于两个发生.
        \item[(8)] $A,B,C$ 中至少有两个发生.
    \end{itemize}
\end{question}
\begin{solution}
    \begin{multicols}{4}
        \begin{itemize}
            \item[(1)] $A\overline{BC}$
            \item[(2)] $AB\overline{C}$
            \item[(3)] $A \cup B \cup C$
            \item[(4)] $ABC$
            \item[(5)] $\overline{ABC}$
            \item[(6)] $\overline{AB}\cup\overline{AC}\cup\overline{BC}$ %$\overline{ABC} + A\overline{BC} + \overline{A}B\overline{C} + \overline{AB}C$ 或者
            \item[(7)] $\overline{A}\cup\overline{B}\cup\overline{C}$ %$A\cup\overline{A} - A \cap B \cap C$ 或者 
            \item[(8)] $AB \cup AC \cup BC$ %$\overline{A}BC + A\overline{B}C + AB\overline{C} + ABC$ 或者 
        \end{itemize}
    \end{multicols}
\end{solution}


\begin{question}{题目3}
    \begin{itemize}
        \item [(1)] 设 $A,B,C$ 是三个事件,且 $P(A) = P(B) = P(C) = \dfrac{1}{4}$,$P(AB) = P(BC) = 0$,$P(AC) = \dfrac{1}{8}$,求 $A,B,C$ 至少有一个发生的概率.

        \item [(2)] 已知 $P(A) = \dfrac{1}{2}$,$P(B) = \dfrac{1}{3}$,$P(C) = \dfrac{1}{5}$,$P(AB) = \dfrac{1}{10}$,$P(AC) = \dfrac{1}{15}$,$P(BC) = \dfrac{1}{20}$,$P(ABC) = \dfrac{1}{30}$,求 $A \cup B$,$\overline{A}\overline{B}$,$A \cup B \cup C$,$\overline{A}\overline{B}\overline{C}$,$\overline{A}\overline{B}C$,$\overline{A} \overline{B}\cup C$ 的概率.

        \item [(3)] 已知 $P(A) = \dfrac{1}{2}$,(i)若 $A,B$ 互不相容,求$P(A\overline{B})$,(ii)若$P(AB) = \dfrac{1}{8}$,求 $P(A\overline{B})$.
    \end{itemize}
\end{question}
\begin{solution}
    (1) 根据多个事件的加法公式
    $$
        \begin{aligned}
            P(A \cup B \cup C)
             & = P(A) + P(B) + P(C) - P(AB) - P(AC) - P(BC) + P(ABC)              \\
             & = \frac{1}{4} + \frac{1}{4} + \frac{1}{4}- 0 - \frac{1}{8} - 0 + 0 \\
             & = \frac{5}{8}.
        \end{aligned}
    $$
    (2) (i) 根据概率的加法公式
    $$
        \begin{aligned}
            P(A \cup B)
            = P(A) + P(B) - P(AB)
            = \frac{1}{2} + \frac{1}{3} - \frac{1}{10}
            = \frac{11}{15}.
        \end{aligned}
    $$
    (ii) 根据事件运算的德摩根律
    $$
        \begin{aligned}
            P\left(\overline{AB}\right)
            = P\left(\overline{A \cup B}\right) = 1-P(A \cup B)
            = 1-\frac{11}{15} = \frac{4}{15}.
        \end{aligned}
    $$
    (iii) 根据多个事件的加法公式
    $$
        \begin{aligned}
            P(A \cup B \cup C)
             & = P(A) + P(B) + P(C) - P(AB) - P(AC) - P(BC) + P(ABC)                                \\
             & = \frac{1}{2} + \frac{1}{3} + \frac{1}{5} - \frac{1}{10}-\frac{1}{20} + \frac{1}{30} \\
             & = \frac{17}{20}.
        \end{aligned}
    $$
    (iv) 根据事件运算的德摩根律
    $$
        \begin{aligned}
            P\left(\overline{ABC}\right)
            = P\left(\overline{A \cup B \cup C}\right)
            = 1 - P(A \cup B \cup C)
            = 1 - \frac{17}{20}
            = \frac{3}{20}.
        \end{aligned}
    $$
    (v) 设样本空间为 $S$,利用 $\overline{C} = S - C$ 转化
    $$
        \begin{aligned}
            P(\overline{AB}C)
            = P\left(\overline{AB}(S-\overline{C})\right)
            = P\left(\overline{AB}\right) - P\left(\overline{ABC}\right)
            = \frac{4}{15} - \frac{3}{20}
            = \frac{7}{60}.
        \end{aligned}
    $$
    (vi) 将 $\overline{AB}$ 视作一个事件,根据加法公式
    $$
        \begin{aligned}
            P(\overline{AB}\cup C)
            = P(\overline{AB}) + P(C) - P(\overline{AB}C)
            = \frac{4}{15} + \frac{1}{5} - \frac{7}{60}
            = \frac{7}{20}.
        \end{aligned}
    $$
    (3) 设样本空间为 $S$
    $$
        P(A\overline{B}) = P(A(S-B)) = P(A - AB) = P(A) - P(AB).
    $$
    (i) 因为 $A,B$ 互不相容,所以 $P(AB) = 0$
    $$
        P(A\overline{B}) = P(A) - P(AB) = \frac{1}{2} - 0 = \frac{1}{2}.
    $$
    (ii) 若 $P(AB) = \dfrac{1}{8}$
    $$
        P(A\overline{B})
        = P(A)-P(AB)
        = \frac{1}{2}-\frac{1}{8}
        = \frac{3}{8}.
    $$
\end{solution}

\begin{question}{题目4}
    设 $A,B$ 是两个事件.
    \begin{itemize}
        \item[(1)] 已知 $A\overline{B} = \overline{A}B$,验证 $A=B$.
        \item[(2)] 验证事件 $A$ 和事件 $B$ 恰有一个发生的概率为 $P(A) + P(B) - 2P(AB)$.
    \end{itemize}
\end{question}
\begin{solution}
    (1) 设样本空间为 $S$
    $$
        \begin{aligned}
            A\overline{B}     & = \overline{A}B     \\
            A\left(S-B\right) & = \left(S-A\right)B \\
            A - AB            & = B - AB            \\
            A                 & = B.
        \end{aligned}
    $$
    (2) 设样本空间为 $S$
    $$
        \begin{aligned}
            P\left(\overline{A}B + A\overline{B}\right)
             & = P(\overline{A}B) + P(A\overline{B}) \\
             & = P[(S-A)B] + P[A(S-B)]               \\
             & = P(B - AB) + P(A - AB)               \\
             & = P(A) + P(B) - 2P(AB).
        \end{aligned}
    $$
\end{solution}

\begin{question}{题目6} 在房间里有 10 个人,分别佩戴从 1 号到 10 号的纪念章,任选 3 人记录其纪念章的号码.
    \begin{itemize}
        \item[(1)] 求最小号码为 5 的概率.
        \item[(2)] 求最大号码为 5 的概率.
    \end{itemize}
\end{question}
\begin{solution}
    (1) 事件$A_1$包含的基本事件数为$C_5^2$,$S$ 中基本事件的总数为 $C_{10}^3$
    $$
        P(A_1) = \frac{C_5^2}{C_{10}^3} = \frac{1}{12}.
    $$
    (2) 事件$A_2$包含的基本事件数为$C_4^2$,$S$中基本事件的总数为 $C_{10}^3$
    $$
        P(A_2) = \frac{C_4^2}{C_{10}^3} = \frac{1}{20}.
    $$
\end{solution}

\begin{question}{题目7}
    某油漆公司发出 17 桶油漆,其中白漆 10 桶、黑漆 4 桶、红漆 3 桶,在搬运中所有标签脱落,交货人随意将这些油漆发给顾客. 问一个订货为 4 桶白漆、3 桶黑漆和 2 桶红漆的顾客,能按所订颜色如数得到订货的概率是多少?
\end{question}
\begin{solution}
    事件 $A$ 包含的基本事件数为 $C_{10}^4 C_4^3 C_3^2$,$S$ 中基本事件的总数为 $C_{17}^9$.
    $$
        P(A) = \frac{C_{10}^4 C_4^3 C_3^2}{C_{17}^9} = \frac{252}{2431}.
    $$
\end{solution}

\begin{question}{补充习题}
    某厂家称一批数量为 1000 件的产品的次品率为 5\%. 现从该批产品中有放回地抽取了 30 件,经检验发现有次品 5 件,问该厂家是否谎报了次品率?
\end{question}
\begin{solution}
    若次品率为厂家声称的 5\%,那么当有放回地抽检 30 件时,出现次品的数量应为
    $$
        30 \times 5\%  = 1.5.
    $$
    而实际抽检出 5 个次品,这说明厂家谎报了次品率.
    \paragraph{方法二} 假设商家如实地上报次品率,那么有放回地抽取 30 件,出现 5 件次品的概率为
    $$
        P = C_{30}^{5}(5\%)^5(1-5\%)^{25} \approx 0.01235.
    $$
    这是一个很低概率的事件,说明厂家谎报了次品率.
\end{solution}

\begin{question}{题目8}
    在 1500 件产品中有 400 件次品、1100 件正品. 任取 200 件.
    \begin{itemize}
        \item [(1)] 求恰有 90 件次品的概率.
        \item [(2)] 求至少有 2 件次品的概率.
    \end{itemize}
\end{question}
\begin{solution}
    (1) 设事件 $A_1$ 为恰有 90 件次品
    $$
        P(A_1) = \frac{C_{400}^{90}C_{1100}^{110}}{C_{1500}^{200}}.
    $$
    (2) 设事件 $A_2$ 为至少有 2 件次品,考虑其对立事件 $\overline{A_2} = \{ \{\text{没有次品}\}, \{\text{只有1件次品}\} \}$
    $$
        P(A_2) = 1 - P(\overline{A_2})
        = 1 - \frac{C_{400}^{0}C_{1100}^{200}}{C_{1500}^{200}} - \frac{C_{400}^{1}C_{1100}^{199}}{C_{1500}^{200}}.
    $$
\end{solution}

\begin{question}{题目11}
    将 3 只球随机地放入 4 个杯子中去,求杯子中球的最大个数分别为 1,2,3 的概率.
\end{question}
\begin{solution}
    设杯中最多有 $i$ 个球为事件 $A_i$
    $$
        P(A_1) = \frac{C_4^1C_3^1C_2^1}{C_4^1C_4^1C_4^1} = \frac{6}{16}.
    $$
    $$
        P(A_2) = \frac{C_3^2C_4^1C_3^1}{C_4^1C_4^1C_4^1} = \frac{9}{16}.
    $$
    $$
        P(A_3) = \frac{C_4^1}{C_4^1C_4^1C_4^1} = \frac{1}{16}.
    $$
\end{solution}

\begin{question}{题目14}
    \begin{itemize}
        \item [(1)] 已知 $P(\overline{A}) = 0.3$,$P(B) = 0.4$,$P(A\overline{B}) = 0.5$,求条件概率 $P(B|A \cup \overline{B})$.
        \item [(2)] 已知 $P(A) = \dfrac{1}{4}$,$P(B|A) = \dfrac{1}{3}$,$P(A|B) = \dfrac{1}{2}$,求 $P(A \cup B)$.
    \end{itemize}
\end{question}
\begin{solution}
    (1) 根据题设条件,有
    $$
        \begin{cases}
            P(\overline{A}) = 1 - P(A) = 0.3,      \\
            P(B) = 1 - P(\overline{B}) = 0.4,      \\
            P(A\overline{B}) = P(A) - P(AB) = 0.5. \\
        \end{cases}
        \implies
        \begin{cases}
            P(A) = 0.7,            \\
            P(\overline{B}) = 0.6, \\
            P(AB) = 0.2.           \\
        \end{cases}
    $$
    再根据条件概率公式和事件运算的分配律,有
    $$
        P(B|A\cup\overline{B})
        = \frac{P(B(A\cup\overline{B}))}{P(A\cup\overline{B})}
        = \frac{P(BA \cup B\overline{B})}{P(A\cup\overline{B})}
        = \frac{P(AB)}{P(A) + P(\overline{B}) - P(A\overline{B})}
        = \frac{1}{4}.
    $$
    (2) 根据题设条件和乘法定理
    $$
        \begin{cases}
            P(AB) = P(B|A)P(A), \\
            P(AB) = P(A|B)P(B). \\
        \end{cases}
        \implies
        \begin{dcases}
            P(B) = \frac{1}{6},   \\
            P(AB) = \frac{1}{12}. \\
        \end{dcases}
    $$
    进一步有
    $$
        P(A \cup B) = P(A) + P(B) - P(AB) = \frac{1}{3}.
    $$
\end{solution}

\begin{question}{题目16}
    据以往资料表明,某 3 口之家,患某种传染病的概率有以下规律:
    $$
        P\{ \text{孩子得病} \} = 0.6,
    $$
    $$
        P\{ \text{母亲得病}|\text{孩子得病} \} = 0.5,
    $$
    $$
        P\{ \text{父亲得病}|\text{母亲及孩子得病} \} = 0.4,
    $$
    求母亲及孩子得病但父亲未得病的概率.
\end{question}
\begin{solution}
    设样本空间为 $S$,事件 $A$ 为孩子得病,事件 $B$ 为母亲得病,事件 $C$ 为父亲得病
    $$
        \begin{dcases}
            P(A) = 0.6,                           \\
            P(B|A) = \frac{P(AB)}{P(A)} = 0.5,    \\
            P(C|BA) = \frac{P(ABC)}{P(AB)} = 0.4. \\
        \end{dcases}
        \implies
        \begin{cases}
            P(AB) = 0.3,    \\
            P(ABC) =  0.12. \\
        \end{cases}
    $$
    进一步有
    $$
        P(AB\overline{C}) = P(AB(S-C)) = P(AB)-P(ABC) = 0.18.
    $$
\end{solution}

\begin{question}{题目37}
    设第一只盒子中装有 3 只蓝球,2 只绿球,2 只白球;第二只盒子中装有 2 只蓝球,3 只绿球,4只白球. 独立地分别在两只盒子中各取一只球.
    \begin{itemize}
        \item [(1)] 求至少有一只蓝球的概率.
        \item [(2)] 求有一只蓝球、一只白球的概率.
        \item [(3)] 已知至少有一只蓝球,求有一只蓝球一只白球的概率.
    \end{itemize}
\end{question}
\begin{solution}
    (1) 设事件 $A$ 为至少有一只蓝球,考虑其对立事件 $\overline{A}$
    $$
        P(A) = 1-P(\overline{A})
        = 1 - \frac{C_4^1}{C_7^1} \frac{C_7^1}{C_9^1}
        = \frac{5}{9}.
    $$
    (2) 设事件 $B$ 为取出一只蓝球和一只白球
    $$
        P(B) = \frac{C_3^1 C_4^1}{C_7^1C_9^1} + \frac{C_2^1C_2^1}{C_7^1C_9^1}
        = \frac{16}{63}.
    $$
    (3) 根据(1)(2)题的结论和条件概率公式
    $$
        P(B|A) = \frac{P(AB)}{P(A)}
        = \frac{16/63}{5/9}
        = \frac{16}{35}.
    $$
\end{solution}